\documentclass[12pt]{report}
\usepackage[utf8]{inputenc}
\usepackage{graphicx}
\graphicspath{ {images/} }

\setlength{\parindent}{4em}
\setlength{\parskip}{1em}
%\renewcommand{\baselinestretch}{2.0}

%subsubsubsection
\newcommand{\subsubsubsection}[1]{\vskip10pt{\noindent\normalsize{\bf {#1}}}\vskip10pt}
%symbols
\input{symbols.tex}




\begin{document}
\bibliographystyle{unsrt}

%TITLE PAGE
\title{
	{VERY MAD PHYSICS GUIDE}\\
	{\large CERN}\\
	{\includegraphics{images.jpg}}
}
\author{I. Tecker, L. Deniau}
\date{18 Feb 2015}
\maketitle
\thispagestyle{empty}


\chapter*{Abstract}
%Abstract goes here
%\begin{flushleft}
\par
MAD-X is a general-purpose tool for charged-particle optics design and studies in alternating-gradient accelerators and beam lines. It can handle medium size to very large accelerators and solves various problems on such machines. MAD-X is the successor of MAD-8 and was specially adapted to the needs of the design of the LHC. The PTC library of E. Forest was also embedded in MAD-X as an addition to better support small and low energy accelerators. 
\par
This document outlines the physical models used in MAD-X. It should help the physicist in understanding the precise function of the program, and appreciate possible limits of validity. 
\par
Comments and corrections from readers are most welcome. They may be sent to the email address: mad@cern.ch
%\end{flushleft}
%end of abstract

\thispagestyle{empty}


%TABLE OF CONTENTS
\newpage
\pagenumbering{roman}
\tableofcontents

%CHAPTERS

\clearpage
\newpage
\pagenumbering{arabic}

\chapter{Conventions}
\input{reference_sys.tex}
\chapter{Hamiltonian}
\input{test_2.tex}

\clearpage
\bibliographystyle{unsrt}   % this means that the order of references
			    % is dtermined by the order in which the
			    % \cite and \nocite commands appear
\bibliography{mybib}

%\bibliographystyle{plain}

\end{document}

