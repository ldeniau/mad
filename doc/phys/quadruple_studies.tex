\documentclass[12pt]{article}
\usepackage[utf8]{inputenc}
\usepackage{graphicx}
\graphicspath{ {images/} }
\usepackage{color}

\setlength{\parindent}{4em}
\setlength{\parskip}{1em}
\renewcommand{\baselinestretch}{1.}

\usepackage[yyyymmdd,hhmmss]{datetime}

\usepackage{amsmath}

\DeclareMathOperator{\Div}{div}
\DeclareMathOperator{\Rot}{rot}


\begin{document}

\title{Quadruple Studies}
\author{Irina Shreyber}
\date{\ddmmyyyydate\today}

\maketitle

\section{The Hamiltonian}
Vector potential for quadruple is: 
\begin{equation}
\pmb{a} = \left(0, 0, -\frac{k_{0}}{s}(x^{2} + y^{2})\right)
\end{equation}

which gives the exact Hamiltonian for quadruple (h = 0):
\begin{align}
H & = \frac{p_{t}}{\beta_{0}}-\sqrt {\left(p_{t}+\frac{1}{\beta_{0}} \right)^{2} - p_{x}^{2} - p_{y}^{2} -\frac{1}{\beta_{0}^{2}\gamma_{0}^{2}}} +\frac{k_{0}}{2}(x^{2}-y^{2}) \nonumber \\
&= \frac{p_{t}}{\beta_{0}}-\sqrt {p_{t}^{2}+2\frac{p_{t}}{\beta_{0}} + 1 - p_{x}^{2} - p_{y}^{2} } +\frac{k_{0}}{2}(x^{2}-y^{2})
\label{eq: Hamiltonian Quad}
\end{align}
 

Then,  with the paraxial approximation by expanding the square root to second order in the dynamical variables we get:

\begin{equation}
\label{eq: Hamiltonian Quad Taylor All}
\tilde{H} = \frac{p_{x}^{2}}{2} + \frac{p_{y}^{2}}{2} +  \frac{p_{t}^{2}}{2\beta_{0}^{2}\gamma_{0}^{2}} +\frac{k_{0}}{2}(x^{2}-y^{2}) 
\end{equation}

In this case we loose the effect of the variation of focusing strength with particle energy in an accelerator beam line ({\it chromaticity}) . To take it into account we can treat the transverse variables $(x, p_{x})$ and $(y, p_{y})$ separately from the longitudinal variables $(z, p_{t})$. This is possible, since if we neglect radiation and certain collective effects, the energy of a particle in a static magnetic field is constant. Therefore, the energy deviation $p_{t}$ of a particle moving through a quadrupole will be constant. We can then expand the Hamiltonian (\ref{eq: Hamiltonian Quad}) to second order in the transverse variables, while maintaining the exact dependence on  $p_{t}$: 

\begin{equation}
\label{eq: Hamiltonian Quad Taylor Part Exact}
\tilde{H} = \frac{p_{x}^{2}}{2(1+\delta)} + \frac{p_{y}^{2}}{2(1+\delta)} + \pmb{ \frac{p_{t}}{\beta_{0}} - (1+\delta)}+\frac{k_{0}}{2}(x^{2}-y^{2}) 
\end{equation}

where 

\begin{equation}
\label{eq: momentum deviation}
(1+\delta) = \sqrt{p_{t}^{2} + 2\frac{p_{t}}{\beta_{0}} + 1},
\end{equation}

and $\delta$ is the momentum deviation. 

If we compare it to Ref[2], Eq.(3)
\begin{equation}
\label{eq: Hamiltonian Andrea}
\tilde{H} = \frac{p_{x}^{2}}{2(1+\delta)} + \frac{p_{y}^{2}}{2(1+\delta)} +  \pmb{\frac{1-\beta_{0}^2}{2\beta_{0}^2}p_{t}^{2}}   +\frac{k_{0}}{2}(x^{2}-y^{2}) 
\end{equation}

we can see that  the term $\frac{1-\beta_{0}^2}{2\beta_{0}^2}p_{t}^{2}$ in (\ref{eq: Hamiltonian Andrea}) is nothing else but the expansion of the term $\frac{p_{t}}{\beta_{0}} - (1+\delta)$  in (\ref{eq: Hamiltonian Quad Taylor Part Exact}) to the second order.

\section{The Equations of Motions}

For both Hamiltonians, (\ref{eq: Hamiltonian Quad Taylor Part Exact}) and (\ref{eq: Hamiltonian Andrea}), the solutions of the equations of motions in the transverse plane are the same:

\begin{align}
\frac{dx}{ds} &= \frac{\partial \tilde{H}}{\partial p_{x}} = \frac{p_{x}}{1+\delta} \nonumber \\
\frac{dp_{x}}{ds} &= -\frac{\partial \tilde{H}}{\partial x} = -k_{0}x \nonumber \\
\frac{dy}{ds} &= \frac{\partial \tilde{H}}{\partial p_{y}} = \frac{p_{y}}{1+\delta} \nonumber \\
\frac{dp_{y}}{ds} &= -\frac{\partial \tilde{H}}{\partial y} = k_{0}y \nonumber \\
\end{align}

whereas the longitudinal ones for $z$ are different due to second order expansion in $p_{t}$. For Hamiltonian (\ref{eq: Hamiltonian Quad Taylor Part Exact}) (we have dropped a constant $\frac{1}{\beta_{0}}$ which has no significance for the dynamics) we have:

\begin{equation}
\label{eq: Motion PT Andy}
\frac{dz}{ds} = \frac{\partial \tilde{H}}{\partial p_{t}} = \pmb{-\frac{1}{(1+\delta)}\left(p_{t} +\frac{1}{\beta_{0}}  \right)}        -      \frac{1}{2}\left(p_{t} + \frac{1}{\beta_{0}}\right)\frac{p_{x}^{2}+ p_{y}^{2}}{(1+\delta)^{3}}     \end{equation}

while for Hamiltonian (\ref{eq: Hamiltonian Andrea}) we get:


\begin{equation}
\label{eq: Motion PT Andrea}
\frac{dz}{ds} = \frac{\partial \tilde{H}}{\partial p_{t}} = \pmb{\frac{1-\beta_{0}^2}{\beta_{0}^2}p_{t}}- \frac{1}{2}\left(p_{t} + \frac{1}{\beta_{0}}\right)\frac{p_{x}^{2}+ p_{y}^{2}}{(1+\delta)^{3}}     
\end{equation}


The equations of motion for $p_{t}$ are obviously the same in both cases:

\begin{equation}
\label{eq: Motion PT}
\frac{dp_{t}}{ds} = -\frac{\partial \tilde{H}}{\partial z} = 0 
\end{equation}


\section{The transfer map}

The transfer map of the thick quadruple is obtained by the integration of the equations of motion over the length $L$. 
We define 

\begin{equation}
\label{eq: strengh}
k = \sqrt{\frac{k_{0}}{1+\delta}}
\end{equation}

\begin{align}
C= cos (\sqrt{k}L), \qquad &\hat{C}= cosh (\sqrt{k}L), \nonumber \\
S= sin (\sqrt{k}L), \qquad &\hat{S}= sinh (\sqrt{k}L);\nonumber
\end{align}


The map for the transverse variable (in both cases) reads:

\begin{align}
%\label{eq: X}
& x(L) = C\cdot x_{0} + \frac{S}{\sqrt{k}(1+\delta)}\cdot  p_{x_{0}} \\
%\label{eq: PX}
&p_{x}(L) =  -\sqrt{k}(1+\delta)S\cdot x_{0} + C\cdot p_{x_{0}} \\
%\label{eq: Y}
& y(L) =  \hat{C}\cdot y_{0} + \frac{\hat{S}}{\sqrt{k}(1+\delta)}\cdot  p_{y_{0}} \\
%\label{eq: PY}
&p_{y}(L)= \sqrt{k}(1+\delta)\hat{S}\cdot y_{0} + \hat{C}\cdot p_{y_{0}} 
\end{align}

For the longitudinal variables $p_{t}$ stays the same in both cases, i.e. 

\begin{equation}
\label{eq: PT}
p_{t}(L) = p_{t_{0}}
\end{equation}

For $z$ in case (\ref{eq: Motion PT Andy}) the solution is:

\begin{align*}
z(L) &=  z_{0} \pmb {-\frac{1}{(1+\delta)}\left(p_{t} +\frac{1}{\beta_{0}}  \right)}  \cdot L \, -\\
&\frac{1}{2}\frac{p_{t} +\frac{1}{\beta_{0}}}{(1+\delta)^{2}} \cdot \Biggl\{  \frac{1}{2}k_{0}\left[ x^{2}\left( L - \frac{C \cdot S}{\sqrt{k}}\right) - y^{2}\left( L - \frac{\hat{C} \cdot \hat{S}}{\sqrt{k}}\right) \right] \, + \\
&\frac{1}{2}\frac{1}{(1+\delta)} \left[ p_{x}^{2}\left( L + \frac{C \cdot S}{\sqrt{k}}\right) + p_{y}^{2}\left( L + \frac{\hat{C} \cdot \hat{S}}{\sqrt{k}}\right) \right] \, -\\ 
& \left[ x\cdot p_{x}\left( 1 - C^{2} \right) + y \cdot p_{y}\left( 1-\hat{C}^{2}\right) \right] \Biggr\}
%\label{eq: Z Andy}
\end{align*}

and in case (\ref{eq: Motion PT Andrea}):

\begin{align*}
z(L) &=  z_{0}+ \pmb {\frac{1-\beta_{0}^2}{\beta_{0}^2}p_{t}} \cdot L + \, \\
&\frac{1}{2}\frac{p_{t} +\frac{1}{\beta_{0}}}{(1+\delta)^{2}} \cdot \Biggl\{  \frac{1}{2}k_{0}\left[ x^{2}\left( L - \frac{C \cdot S}{\sqrt{k}}\right) - y^{2}\left( L - \frac{\hat{C} \cdot \hat{S}}{\sqrt{k}}\right) \right] \, + \\
&\frac{1}{2}\frac{1}{(1+\delta)} \left[ p_{x}^{2}\left( L + \frac{C \cdot S}{\sqrt{k}}\right) + p_{y}^{2}\left( L + \frac{\hat{C} \cdot \hat{S}}{\sqrt{k}}\right) \right] \, -\\ 
& \left[ x\cdot p_{x}\left( 1 - C^{2} \right) + y \cdot p_{y}\left( 1-\hat{C}^{2}\right) \right] \Biggr\}
%\label{eq:Z Andrea}
\end{align*}

As one can see, the only difference is the coefficient in front of the linear term which is coming from the choice of the "quasi"-exact solutions for $p_{t}$ or the second-order approximation. It makes sense to replace second-order expansion in $p_{t}$ by the exact solution for the new MAD (?). In this particular case we can easily have the "exact" solution without adding any complication to the calculation.


{\it
\pmb{Reference:}

1. A.Wolski, Beam Dynamics in HEP Accelerators, chapter 3, p.101-105 


2. A. Latina, Implementation of a Thick Quadrupole in the MAD-X tracking module
}
\end{document}
